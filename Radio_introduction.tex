%!TEX root = RadioPaper_Main.tex
Online music services such as Spotify, Pandora, Google Play Music and Microsoft's Groove serve as a major growth engine for today's music industry. A key experience is the ability to stream automatically generated playlists based on some criteria chosen by the user.
%Scenarios such as genre, artist and track-based radio are tasked with creating audio playlists which suit user tastes and can be several hours long, based on limited explicit input from the user. 
This paper considers the problem of automatically generating personalized music playlists based on queries containing a ``seed'' artist and the listener's user ID \SBE{I feel that this is too repetitive across the paper.}\GL{IMO this is preferable to having the reader be lost looking for definitions in the intro while reading the model}. We describe a solution designed for Microsoft's Groove internet radio \SBE{removed quotation marks from radio to be consistent with title} service, which serves playlists to millions of users world wide. 
%This model was implemented and compared to Groove's previous playlist solution using an online experiment showing an increase in users' listening time and a reduction in the number of skipped tracks.

In recommender systems research, collaborative filtering approaches such as matrix factorization are often used to learn relations between users and items \cite{KorenBV09}. 
The playlist generation task is fundamentally different as it requires learning a coherent sequence to allow smooth track transitions.
Central to the approach taken in this research is the idea of estimating the relatedness of a ``candidate'' track to the previously selected tracks already in the playlist.
%A Naive approach might select one particular type of similarity (e.g. tracks from the same genre or tracks from the same era) and return all appropriate tracks according to this measure.  
%Moreover, matrix factorization approaches are based on learning latent features using on one type of information source typically explicit ratings or implicit usage patterns. 
Such relatedness can depend on multiple information sources, such as meta-data and domain semantics, the acoustic audio signal,  and popularity patterns, as well as information extracted using collaborative filtering techniques.  
%Meta-data similarity is defined by the semantics of two tracks. %(e.g. both tracks are recorded by the same artist). 
%Usage similarity is defined by the consumption patterns of two tracks (e.g. users who listened to one also listened to the other) and can be learned using collaborative filtering methods such as matrix factorization. Acoustic similarity is extracted by digital signal processing on the audio signal.% of two tracks (e.g. they both contain drums and piano). 
%Beyond similarity, consumption patterns of commercial music are also inherently trendy. That is, popular tracks may be a welcome addition to a playlist regardless of their similarity to other tracks in the list. 

The multiplicity of useful signals has motivated recent works  \cite{personalized_playlist_schedl,McFee_multi_similarities} to combine several information sources in order to model playlists. % and provide a more robust estimate of relatedness.  
While it is known that all these factors play a role in the construction of a quality playlist, a key distinction of Groove's model is the idea that the importance of each of these information sources varies depending on the specific seed artist. For example, when composing a playlist for a \textit{Jazz} artist such as \textit{John Coltrane}, the importance of acoustic similarity features may be high. In contrast, in the case of a \textit{Contemporary Pop} artist, such as \textit{Lady Gaga}, features based on collaborative filtering may be more important. In Section~\ref{sec:experiments}, evaluations are provided in support of this assumption.
 
Another key element of the model in this paper is personalization based on the user. For example, a particular user may prefer strictly popular tracks, while another prefers more esoteric content. %Yet another user may prefer music from a particular genre or artist. 
Our method provides for modeling user specific preferences via a personalization component.


%This taxonomy allows to capture variance in the importance of features at different levels of the hierarchy. For example, the variance between subgenres in the ``World Music'' genre could possibly be much greater than other genres such as Rock music, due to the eclectic nature of the former. Thus, our model would learn the necessary  sub-genre or even artist level distinction between the factors that should be applied to construct the playlist.

The model in this paper is designed to support any artist in Groove's catalog, far beyond the small number of artists that dominate the lion's share of user listening patterns. The distribution of artists in the catalog contains a long tail of less popular artists for which insufficient information exists to learn artist specific parameters. Therefore, the Groove model also leverages the hierarchical music domain taxonomy of genres, sub-genres and artists. %I omitted album and tracks since these are not parts of the hierarchy in our model. Feel free to revert. 
When particular artists or possibly even sub-genres are underrepresented in the data, the Groove model can still allow prediction by ``borrowing'' information from sibling nodes sharing the same parent in the taxonomy. %Similarly, our model also affords  ``cold users" the benefit of the learned domain taxonomy parameters, allowing construction of a quality playlist upon their very first launch of the service. 

%This  paper proposes a novel machine learning approach encapsulating the above attributes. 
The Groove model casts playlist construction as a classification problem: predicting the next track given the seed artist, the user ID, and previously played tracks. 
As such, a hierarchical Bayesian model is suggested to perform playlist prediction using logistic regression. 
%This approach is essentially different from most collaborative filtering algorithms in which there is no 
%Specifically, the four-level hierarchy of global preferences, genres, sub-genres, and artists. 
Parameter learning follows the variational Bayes approach of approximating the posterior distribution using a probability distribution which factorizes over the parameters. 
%We evaluate our approach on two datasets, each representing a slightly different type of signal. The Groove music dataset, represents explicit skip data. That is, negative examples in this dataset are explicitly provided by users.
%\todo{We should be careful of the ``dataset shift'' issue, where explicitly provided negative data are a function of our
%current model. For the sake of the reviewers, we should have an answer about the feedback loop and biases, as the negatives are not i.i.d.\ldots}\GL{ I tried to adress this in the practical considerations section, do you think more is needed on this point?}
%The 30Music dataset \cite{DBLP:conf/recsys/TurrinQCPC15}, is a dataset of user compiled playlists. In this dataset, the negatives are implied. They are sampled from those tracks that do not appear \GL{not sure what 'together' means here} on a user defined playlist. 
%In this type of signal, the semantic hierarchy is able to perform nearly as well as the personalization.
The evaluation of this model demonstrates the contribution of the hierarchical domain taxonomy and the personalization component to the prediction task at hand. 
%We are able to achieve AUC of 0.9 and 0.86 for the two datasets, respectively. We also provide an analysis of the benefit afforded by considering different sets (and combinations of) similarity feature types, as well as an analysis of the robustness of our approach to user and item ``coldness" \SBE{need to be consistent in placing this term in quotation marks} \GL{we need to make sure to synchronize this passage in case any changes are made to the experiments section}. 
Beyond this evaluation, the performance of a production variant of the approach described in this paper was validated in an online A/B experiment. Our online experimentations (not discussed in this paper) showed an increase in user listening time and a reduction in the number of skipped tracks relative to any alternative playlist algorithm. 

\GLw{Added organization.. can be removed again if we are short on space}
%Our paper is organized as follows: Section \ref{sec:Related} discusses relevant related work. Section \ref{sec:OurModel} motivates our approach, discusses the specification of our model and explains the  algorithm we apply to learn the parameters from data. Section \ref{sec:Features} describes the features we use to encode  playlist context. Section \ref{sec:experiments} gives the details of our experimental evaluation. Section \ref{sec:practicalConsiderations} describes additional modifications that allow the proposed approach to work in large-scale scenarios. Finally, section \ref{sec:conclusion} summarizes the paper.


