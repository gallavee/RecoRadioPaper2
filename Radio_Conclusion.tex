This work describes a model for playlist generation designed for Microsoft's Groove music service. The model incorporates per-artist parameters in order to capture the unique characteristics of an artist's playlists. The domain taxonomy of genres, sub-genres and artists is utilized in order to allow training examples from one artist to inform predictions of other related artists. 
%Hence, the model offers good predictions even when a playlist is requested for a ``seed'' artist for whom little (or even no) historical information is available. 
Furthermore, the proposed model is endowed with the capacity to capture particular user preferences for those users who are frequent playlist requesters, enabling a personalized playlist experience. %For ``cold'' users who are using the system for the first few times, the semantics of the ``seed'' artist enable the model to provide a good playlist without relying only on user information.
A variational inference learning algorithm is applied and evaluations are provided to justify and showcase the importance of each of the model's properties from above. 

%a detailed description of the update steps of each parameter is provided in Section~\ref{sec:OurModel}. %This algorithm is run iteratively, until the convergence of a bound on the variational free energy.
%The evaluations in Section~\ref{sec:experiments} show the benefit of using the domain taxonomy as well as the personalization components for the playlist generation task.% Finally, an online experiment is described showing a significant increase in user listening time and a significant reduction in skipped tracks. %Some practical considerations for implementing the proposed approach at scale on a real-world system are discussed in the final section.

