This paper describes an algorithm designed for Microsoft's Groove music service, which serves millions of users world wide.  We consider the problem of automatically generating personalized music playlists based on queries containing a ``seed'' artist and the listener's user ID. 
Playlist generation may be informed by a number of information sources including: user specific listening patterns, knowledge of the domain encoded in a taxonomy, acoustic features of audio tracks, and overall popularity of tracks and artists. The importance assigned to each of these information sources may vary depending on the specific combination of user and seed artist.

The paper presents a method based on a variational Bayes solution for learning the parameters of a model containing a four-level hierarchy of global preferences, genres, sub-genres and artists. The proposed model further incorporates a personalization component for user-specific preferences.
Empirical evaluations on both private and public datasets demonstrate the effectiveness of the algorithm and showcase the contribution of each of its components.\GLw{Moved to intro: An online A/B experiment, comparing this algorithm to Groove's previous playlist algorithm, shows an increase in user listening time as well as a reduction in the number of skipped tracks.}

%The problem of automatic playlist generation is at the core of the intersection between recommender systems and the music domain. 
%Microsoft's Groove music service offers automatic playlists based on a ``seed'' artist and listener history, serving millions of users world-wide. 
\SBE{I want to propose a more engaging first paragraph: The problem of automatic playlist generation is at the core of the intersection between recommender systems and the music domain. Microsoft's Groove music service offers automatic playlists based on a ``seed'' artist and listener history, serving millions of users world-wide. etc.}\GL{Looks good to me, lets integrate it and start converging on a final version} \noam{please no. I really liked the old version. what can be more engaging than a description of a real world system? Besides, the MIR community is small and sometimes looked down upon by outsiders due their high acceptance rate policy.}
